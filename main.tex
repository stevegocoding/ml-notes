\documentclass{ctexart}

\usepackage{amsmath, amssymb}

\DeclareMathOperator{\E}{\mathbb{E}}

\begin{document}

\tableofcontents

%%%%%%%%%%%%%%%%%%%%%%%%%%%%%%%%%%%%%%%
% 信息论 
%%%%%%%%%%%%%%%%%%%%%%%%%%%%%%%%%%%%%%%

\section{ Information Theory }

信息学主要是用来quantify在信号中信息的含量。在ML中, 信息学主要是用来描述随机分布或者量化不同随机分之间的相似性。

\subsection{Problem Setting}

\subsection{ Important Quantities }

1)信息熵:编码方案完美时,最短平均编码长度的是多少。2)交叉熵:编码方案不一定完美时(由于对概率分布的估计不一定正确),平均编码长度的是多少。              平均编码长度 = 最短平均编码长度 + 一个增量3)相对熵:编码方案不一定完美时,平均编码长度相对于最小值的增加值。(即上面那个增量)

% 自信息
\paragraph{Self Information}

\begin{itemize}
    \item Likely events should have low information content. If events are guaranteed to happen, they should have no information content.
    
    \item Less likely events should have higher information content.
\end{itemize}

The self-information of an random event \( x \): 
\begin{equation}
I(x) = - \log P(x)
\end{equation}

\( P(x) \) is the probability of the random event \( x \).

% 熵
\paragraph{Shannon Entropy}

Self-information deals only with single event of a random variable. Shannon entropy quantify the amount of uncertainty in an entire probability distribution.

TODO: Measures the randomness of a distribution.

\begin{equation}
    H( \text{x} ) = \E_{\text{x} \sim P} I(x) = -\E_{\text{x} \sim P} \log P(x)
\end{equation}

\text{x} is the random variable, \( x \) is the values of \text{x}.

In other words, Shannon entropy of a distribution is the expected amount of information in an event drawn from that distribution.

When \text{x} is a continuous random variable, Shannon entropy is known as the \textbf{differential entropy}.

% KL 散度
\paragraph{KL Divergence}


In the context of decision tree, KL-Divergence is also known as \textbf{Information Gain} or \textbf{Relative Entropy}.






\end{document}